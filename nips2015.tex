\documentclass{article} % For LaTeX2e
\usepackage{nips15submit_e,times}
\usepackage{hyperref}
\usepackage{url}
\usepackage{amssymb}
\usepackage{booktabs}

%\documentstyle[nips14submit_09,times,art10]{article} % For LaTeX 2.09

\title{Accelerating Multimodal Sequence Retrieval with Convolutional Networks}


\author{
Colin Raffel
LabROSA
Columbia University
New York, NY 10027
\texttt{craffel@gmail.com}
\And
Daniel P.~W.~Ellis
LabROSA
Columbia University
New York, NY 10027
\texttt{dpwe@ee.columbia.edu}
}

\newcommand{\fix}{\marginpar{FIX}}
\newcommand{\new}{\marginpar{NEW}}

%\nipsfinalcopy % Uncomment for camera-ready version

\begin{document}

\maketitle

\begin{abstract}
Given a large database of sequential data, a natural problem is to find the entry in the database which is most similar to a query sequence.
Warping-based similarity metrics such as the Dynamic Time Warping (DTW) distance can be prohibitively expensive when the sequences are long and/or high-dimensional.
To mitigate these issues, \cite{raffel2015large} utilizes a convolutional network to map sequences of feature vectors to downsampled sequences of binary vectors.
On the task of matching synthetic renditions of pieces of music to a large database of audio recordings of songs, this approach was able to efficiently discard 99\% of the database with high confidence.
We extend this approach to the multimodal setting where rather than synthetic renditions a matrix representation of the piece's score is used instead, demonstrating that this approach is adaptable to the underlying representation.
\end{abstract}

\bibliographystyle{unsrt}
\small
\bibliography{refs}

\end{document}
